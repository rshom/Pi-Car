%% Generated by Sphinx.
\def\sphinxdocclass{report}
\documentclass[letterpaper,10pt,english]{sphinxmanual}
\ifdefined\pdfpxdimen
   \let\sphinxpxdimen\pdfpxdimen\else\newdimen\sphinxpxdimen
\fi \sphinxpxdimen=.75bp\relax

\PassOptionsToPackage{warn}{textcomp}
\usepackage[utf8]{inputenc}
\ifdefined\DeclareUnicodeCharacter
% support both utf8 and utf8x syntaxes
  \ifdefined\DeclareUnicodeCharacterAsOptional
    \def\sphinxDUC#1{\DeclareUnicodeCharacter{"#1}}
  \else
    \let\sphinxDUC\DeclareUnicodeCharacter
  \fi
  \sphinxDUC{00A0}{\nobreakspace}
  \sphinxDUC{2500}{\sphinxunichar{2500}}
  \sphinxDUC{2502}{\sphinxunichar{2502}}
  \sphinxDUC{2514}{\sphinxunichar{2514}}
  \sphinxDUC{251C}{\sphinxunichar{251C}}
  \sphinxDUC{2572}{\textbackslash}
\fi
\usepackage{cmap}
\usepackage[T1]{fontenc}
\usepackage{amsmath,amssymb,amstext}
\usepackage{babel}



\usepackage{times}
\expandafter\ifx\csname T@LGR\endcsname\relax
\else
% LGR was declared as font encoding
  \substitutefont{LGR}{\rmdefault}{cmr}
  \substitutefont{LGR}{\sfdefault}{cmss}
  \substitutefont{LGR}{\ttdefault}{cmtt}
\fi
\expandafter\ifx\csname T@X2\endcsname\relax
  \expandafter\ifx\csname T@T2A\endcsname\relax
  \else
  % T2A was declared as font encoding
    \substitutefont{T2A}{\rmdefault}{cmr}
    \substitutefont{T2A}{\sfdefault}{cmss}
    \substitutefont{T2A}{\ttdefault}{cmtt}
  \fi
\else
% X2 was declared as font encoding
  \substitutefont{X2}{\rmdefault}{cmr}
  \substitutefont{X2}{\sfdefault}{cmss}
  \substitutefont{X2}{\ttdefault}{cmtt}
\fi


\usepackage[Bjarne]{fncychap}
\usepackage{sphinx}

\fvset{fontsize=\small}
\usepackage{geometry}

% Include hyperref last.
\usepackage{hyperref}
% Fix anchor placement for figures with captions.
\usepackage{hypcap}% it must be loaded after hyperref.
% Set up styles of URL: it should be placed after hyperref.
\urlstyle{same}
\addto\captionsenglish{\renewcommand{\contentsname}{Contents:}}

\usepackage{sphinxmessages}
\setcounter{tocdepth}{2}



\title{PiCar}
\date{Dec 08, 2019}
\release{0.0.1}
\author{Russell Shomberg}
\newcommand{\sphinxlogo}{\vbox{}}
\renewcommand{\releasename}{Release}
\makeindex
\begin{document}

\pagestyle{empty}
\sphinxmaketitle
\pagestyle{plain}
\sphinxtableofcontents
\pagestyle{normal}
\phantomsection\label{\detokenize{index::doc}}



\chapter{The picar API reference}
\label{\detokenize{index:module-picar}}\label{\detokenize{index:the-picar-api-reference}}\index{picar (module)@\spxentry{picar}\spxextra{module}}\index{Camera (class in picar)@\spxentry{Camera}\spxextra{class in picar}}

\begin{fulllineitems}
\phantomsection\label{\detokenize{index:picar.Camera}}\pysiglinewithargsret{\sphinxbfcode{\sphinxupquote{class }}\sphinxcode{\sphinxupquote{picar.}}\sphinxbfcode{\sphinxupquote{Camera}}}{\emph{resolution=640.48}}{}
Camera class class which can read or stream over a socket
\index{close() (picar.Camera method)@\spxentry{close()}\spxextra{picar.Camera method}}

\begin{fulllineitems}
\phantomsection\label{\detokenize{index:picar.Camera.close}}\pysiglinewithargsret{\sphinxbfcode{\sphinxupquote{close}}}{}{}
Close camera nicely

\end{fulllineitems}

\index{read() (picar.Camera method)@\spxentry{read()}\spxextra{picar.Camera method}}

\begin{fulllineitems}
\phantomsection\label{\detokenize{index:picar.Camera.read}}\pysiglinewithargsret{\sphinxbfcode{\sphinxupquote{read}}}{}{}
Return frame same as cv2.read()

\end{fulllineitems}

\index{stream() (picar.Camera method)@\spxentry{stream()}\spxextra{picar.Camera method}}

\begin{fulllineitems}
\phantomsection\label{\detokenize{index:picar.Camera.stream}}\pysiglinewithargsret{\sphinxbfcode{\sphinxupquote{stream}}}{\emph{sock}}{}
Continuous capture and send over network

\end{fulllineitems}


\end{fulllineitems}

\index{LineSensor (class in picar)@\spxentry{LineSensor}\spxextra{class in picar}}

\begin{fulllineitems}
\phantomsection\label{\detokenize{index:picar.LineSensor}}\pysiglinewithargsret{\sphinxbfcode{\sphinxupquote{class }}\sphinxcode{\sphinxupquote{picar.}}\sphinxbfcode{\sphinxupquote{LineSensor}}}{\emph{pin\_middle=16}, \emph{pin\_left=19}, \emph{pin\_right=20}, \emph{blackLine=True}}{}
Line sensor consisting of 3 elements in a linear array

\end{fulllineitems}

\index{Motor (class in picar)@\spxentry{Motor}\spxextra{class in picar}}

\begin{fulllineitems}
\phantomsection\label{\detokenize{index:picar.Motor}}\pysiglinewithargsret{\sphinxbfcode{\sphinxupquote{class }}\sphinxcode{\sphinxupquote{picar.}}\sphinxbfcode{\sphinxupquote{Motor}}}{\emph{en}, \emph{pin1}, \emph{pin2}}{}
Class for controlling the DC motor
\index{coast() (picar.Motor method)@\spxentry{coast()}\spxextra{picar.Motor method}}

\begin{fulllineitems}
\phantomsection\label{\detokenize{index:picar.Motor.coast}}\pysiglinewithargsret{\sphinxbfcode{\sphinxupquote{coast}}}{}{}
Releases motor to coast

\end{fulllineitems}

\index{drive() (picar.Motor method)@\spxentry{drive()}\spxextra{picar.Motor method}}

\begin{fulllineitems}
\phantomsection\label{\detokenize{index:picar.Motor.drive}}\pysiglinewithargsret{\sphinxbfcode{\sphinxupquote{drive}}}{\emph{speed}}{}
checks direction and moves

\end{fulllineitems}

\index{forward() (picar.Motor method)@\spxentry{forward()}\spxextra{picar.Motor method}}

\begin{fulllineitems}
\phantomsection\label{\detokenize{index:picar.Motor.forward}}\pysiglinewithargsret{\sphinxbfcode{\sphinxupquote{forward}}}{\emph{speed=100}}{}
Drives forward

\end{fulllineitems}

\index{reverse() (picar.Motor method)@\spxentry{reverse()}\spxextra{picar.Motor method}}

\begin{fulllineitems}
\phantomsection\label{\detokenize{index:picar.Motor.reverse}}\pysiglinewithargsret{\sphinxbfcode{\sphinxupquote{reverse}}}{\emph{speed=100}}{}
Drives backwards

\end{fulllineitems}

\index{stop() (picar.Motor method)@\spxentry{stop()}\spxextra{picar.Motor method}}

\begin{fulllineitems}
\phantomsection\label{\detokenize{index:picar.Motor.stop}}\pysiglinewithargsret{\sphinxbfcode{\sphinxupquote{stop}}}{}{}
Stops motor after pulling in the opposite direction to hard stop

\end{fulllineitems}


\end{fulllineitems}

\index{PiCar (class in picar)@\spxentry{PiCar}\spxextra{class in picar}}

\begin{fulllineitems}
\phantomsection\label{\detokenize{index:picar.PiCar}}\pysigline{\sphinxbfcode{\sphinxupquote{class }}\sphinxcode{\sphinxupquote{picar.}}\sphinxbfcode{\sphinxupquote{PiCar}}}
Provides an interface to control the car.

Upon construction, this class initializes all controls and sensors.
For controls, the car has a motor, a turning servo, and 2 servos controlling the head.
For sensors, the car has a line sensor, a sonar, and a camera.
The constructor expects all the peripherals to be plugged into the pi in a specific manner which can only be changed by directly changing the code of the constructor.
\index{all\_ahead() (picar.PiCar method)@\spxentry{all\_ahead()}\spxextra{picar.PiCar method}}

\begin{fulllineitems}
\phantomsection\label{\detokenize{index:picar.PiCar.all_ahead}}\pysiglinewithargsret{\sphinxbfcode{\sphinxupquote{all\_ahead}}}{}{}
Bring all controls to forward

\end{fulllineitems}

\index{all\_stop() (picar.PiCar method)@\spxentry{all\_stop()}\spxextra{picar.PiCar method}}

\begin{fulllineitems}
\phantomsection\label{\detokenize{index:picar.PiCar.all_stop}}\pysiglinewithargsret{\sphinxbfcode{\sphinxupquote{all\_stop}}}{}{}
Stop the car and face forward. Exit any current mode

\end{fulllineitems}

\index{ebrake() (picar.PiCar method)@\spxentry{ebrake()}\spxextra{picar.PiCar method}}

\begin{fulllineitems}
\phantomsection\label{\detokenize{index:picar.PiCar.ebrake}}\pysiglinewithargsret{\sphinxbfcode{\sphinxupquote{ebrake}}}{\emph{dir=1}}{}
Hard brake by reversing for a second

Change dir to -1 to use ebrake when reversed

\end{fulllineitems}

\index{follow\_line() (picar.PiCar method)@\spxentry{follow\_line()}\spxextra{picar.PiCar method}}

\begin{fulllineitems}
\phantomsection\label{\detokenize{index:picar.PiCar.follow_line}}\pysiglinewithargsret{\sphinxbfcode{\sphinxupquote{follow\_line}}}{\emph{darkLine=False}, \emph{speed=1}, \emph{gain=(10}, \emph{60}, \emph{0)}, \emph{nHist=100}, \emph{maxAng=30}}{}
Look for a line and drive along following it

The follow line is a generator needs to be continuously called in order to continue following the line.

\end{fulllineitems}

\index{pulse() (picar.PiCar method)@\spxentry{pulse()}\spxextra{picar.PiCar method}}

\begin{fulllineitems}
\phantomsection\label{\detokenize{index:picar.PiCar.pulse}}\pysiglinewithargsret{\sphinxbfcode{\sphinxupquote{pulse}}}{\emph{runTime}, \emph{coastTime}}{}
Continuously pulse the motor.

Generally, this method should be called as a thread to run in the background. Otherwise it will block the program from running.

\end{fulllineitems}

\index{run\_cmd() (picar.PiCar method)@\spxentry{run\_cmd()}\spxextra{picar.PiCar method}}

\begin{fulllineitems}
\phantomsection\label{\detokenize{index:picar.PiCar.run_cmd}}\pysiglinewithargsret{\sphinxbfcode{\sphinxupquote{run\_cmd}}}{\emph{cmd}, \emph{arg=''}}{}
Run a preset command

\end{fulllineitems}

\index{sonar\_scan() (picar.PiCar method)@\spxentry{sonar\_scan()}\spxextra{picar.PiCar method}}

\begin{fulllineitems}
\phantomsection\label{\detokenize{index:picar.PiCar.sonar_scan}}\pysiglinewithargsret{\sphinxbfcode{\sphinxupquote{sonar\_scan}}}{\emph{distance=1}, \emph{scanSpeed=1}, \emph{tiltAngle=0}}{}
Measure distances across full range of sonar

Car will first look all the way to the left. It will then slowly turn all the way to the right, while making a sonar measurement at each angle.
distance determines how long the ping should wait for a response.
scanSpeed determines how far apart the pings are angularly. A slower scanSpeed means more pings and denser set of results. scanSpeed must be an integer greater than 0.

sonar\_scan() returns a list of distances and a second list of corresponding angles.

\end{fulllineitems}

\index{track\_object() (picar.PiCar method)@\spxentry{track\_object()}\spxextra{picar.PiCar method}}

\begin{fulllineitems}
\phantomsection\label{\detokenize{index:picar.PiCar.track_object}}\pysiglinewithargsret{\sphinxbfcode{\sphinxupquote{track\_object}}}{}{}
Keep an object in view and follow it

\end{fulllineitems}


\end{fulllineitems}

\index{Servo (class in picar)@\spxentry{Servo}\spxextra{class in picar}}

\begin{fulllineitems}
\phantomsection\label{\detokenize{index:picar.Servo}}\pysiglinewithargsret{\sphinxbfcode{\sphinxupquote{class }}\sphinxcode{\sphinxupquote{picar.}}\sphinxbfcode{\sphinxupquote{Servo}}}{\emph{pin}, \emph{pwmMin}, \emph{pwmMax}, \emph{pwmCenter}, \emph{angMin}, \emph{angMax}, \emph{angCenter}}{}
Class defining a servo
\index{center() (picar.Servo method)@\spxentry{center()}\spxextra{picar.Servo method}}

\begin{fulllineitems}
\phantomsection\label{\detokenize{index:picar.Servo.center}}\pysiglinewithargsret{\sphinxbfcode{\sphinxupquote{center}}}{}{}
Go to center

\end{fulllineitems}

\index{goto() (picar.Servo method)@\spxentry{goto()}\spxextra{picar.Servo method}}

\begin{fulllineitems}
\phantomsection\label{\detokenize{index:picar.Servo.goto}}\pysiglinewithargsret{\sphinxbfcode{\sphinxupquote{goto}}}{\emph{angle}}{}
Rotate to angle

\end{fulllineitems}

\index{max() (picar.Servo method)@\spxentry{max()}\spxextra{picar.Servo method}}

\begin{fulllineitems}
\phantomsection\label{\detokenize{index:picar.Servo.max}}\pysiglinewithargsret{\sphinxbfcode{\sphinxupquote{max}}}{}{}
Go to max

\end{fulllineitems}

\index{min() (picar.Servo method)@\spxentry{min()}\spxextra{picar.Servo method}}

\begin{fulllineitems}
\phantomsection\label{\detokenize{index:picar.Servo.min}}\pysiglinewithargsret{\sphinxbfcode{\sphinxupquote{min}}}{}{}
Go to min

\end{fulllineitems}

\index{rotate() (picar.Servo method)@\spxentry{rotate()}\spxextra{picar.Servo method}}

\begin{fulllineitems}
\phantomsection\label{\detokenize{index:picar.Servo.rotate}}\pysiglinewithargsret{\sphinxbfcode{\sphinxupquote{rotate}}}{\emph{angle}}{}
Rotate an incriment

\end{fulllineitems}


\end{fulllineitems}



\chapter{The “server” module}
\label{\detokenize{index:module-server}}\label{\detokenize{index:the-server-module}}\index{server (module)@\spxentry{server}\spxextra{module}}\index{FootageStream (class in server)@\spxentry{FootageStream}\spxextra{class in server}}

\begin{fulllineitems}
\phantomsection\label{\detokenize{index:server.FootageStream}}\pysiglinewithargsret{\sphinxbfcode{\sphinxupquote{class }}\sphinxcode{\sphinxupquote{server.}}\sphinxbfcode{\sphinxupquote{FootageStream}}}{\emph{camera}, \emph{addr}, \emph{port=5555}}{}
Continuously serves requests for the latest frame
\index{run() (server.FootageStream method)@\spxentry{run()}\spxextra{server.FootageStream method}}

\begin{fulllineitems}
\phantomsection\label{\detokenize{index:server.FootageStream.run}}\pysiglinewithargsret{\sphinxbfcode{\sphinxupquote{run}}}{}{}
Send images to the connected socket

\end{fulllineitems}


\end{fulllineitems}

\index{Reciever (class in server)@\spxentry{Reciever}\spxextra{class in server}}

\begin{fulllineitems}
\phantomsection\label{\detokenize{index:server.Reciever}}\pysiglinewithargsret{\sphinxbfcode{\sphinxupquote{class }}\sphinxcode{\sphinxupquote{server.}}\sphinxbfcode{\sphinxupquote{Reciever}}}{\emph{addr=''}, \emph{port=5000}}{}~\index{run() (server.Reciever method)@\spxentry{run()}\spxextra{server.Reciever method}}

\begin{fulllineitems}
\phantomsection\label{\detokenize{index:server.Reciever.run}}\pysiglinewithargsret{\sphinxbfcode{\sphinxupquote{run}}}{}{}
Method representing the thread’s activity.

You may override this method in a subclass. The standard run() method
invokes the callable object passed to the object’s constructor as the
target argument, if any, with sequential and keyword arguments taken
from the args and kwargs arguments, respectively.

\end{fulllineitems}


\end{fulllineitems}

\index{main() (in module server)@\spxentry{main()}\spxextra{in module server}}

\begin{fulllineitems}
\phantomsection\label{\detokenize{index:server.main}}\pysiglinewithargsret{\sphinxcode{\sphinxupquote{server.}}\sphinxbfcode{\sphinxupquote{main}}}{}{}
Start the picar and wait for commands

\end{fulllineitems}



\chapter{The “client” module}
\label{\detokenize{index:module-client}}\label{\detokenize{index:the-client-module}}\index{client (module)@\spxentry{client}\spxextra{module}}\index{Keyboard (class in client)@\spxentry{Keyboard}\spxextra{class in client}}

\begin{fulllineitems}
\phantomsection\label{\detokenize{index:client.Keyboard}}\pysiglinewithargsret{\sphinxbfcode{\sphinxupquote{class }}\sphinxcode{\sphinxupquote{client.}}\sphinxbfcode{\sphinxupquote{Keyboard}}}{\emph{transmitter}}{}
Keystroke commands that can be sent over the transmitter
\index{run() (client.Keyboard method)@\spxentry{run()}\spxextra{client.Keyboard method}}

\begin{fulllineitems}
\phantomsection\label{\detokenize{index:client.Keyboard.run}}\pysiglinewithargsret{\sphinxbfcode{\sphinxupquote{run}}}{}{}
Background thread listening to for keyboard events

\end{fulllineitems}

\index{run\_cmd() (client.Keyboard method)@\spxentry{run\_cmd()}\spxextra{client.Keyboard method}}

\begin{fulllineitems}
\phantomsection\label{\detokenize{index:client.Keyboard.run_cmd}}\pysiglinewithargsret{\sphinxbfcode{\sphinxupquote{run\_cmd}}}{\emph{cmd}}{}
Send a command to picar reciever

\end{fulllineitems}


\end{fulllineitems}

\index{LiveFeed (class in client)@\spxentry{LiveFeed}\spxextra{class in client}}

\begin{fulllineitems}
\phantomsection\label{\detokenize{index:client.LiveFeed}}\pysiglinewithargsret{\sphinxbfcode{\sphinxupquote{class }}\sphinxcode{\sphinxupquote{client.}}\sphinxbfcode{\sphinxupquote{LiveFeed}}}{\emph{addr='192.168.0.212'}, \emph{port=5555}, \emph{threaded=False}}{}
Thread that receives frames from server and makes them available to other client threads.
\index{read() (client.LiveFeed method)@\spxentry{read()}\spxextra{client.LiveFeed method}}

\begin{fulllineitems}
\phantomsection\label{\detokenize{index:client.LiveFeed.read}}\pysiglinewithargsret{\sphinxbfcode{\sphinxupquote{read}}}{}{}
Read a single frame from server

\end{fulllineitems}

\index{run() (client.LiveFeed method)@\spxentry{run()}\spxextra{client.LiveFeed method}}

\begin{fulllineitems}
\phantomsection\label{\detokenize{index:client.LiveFeed.run}}\pysiglinewithargsret{\sphinxbfcode{\sphinxupquote{run}}}{}{}
Continuously read frames and update in background

\end{fulllineitems}


\end{fulllineitems}

\index{ObjectTracker (class in client)@\spxentry{ObjectTracker}\spxextra{class in client}}

\begin{fulllineitems}
\phantomsection\label{\detokenize{index:client.ObjectTracker}}\pysiglinewithargsret{\sphinxbfcode{\sphinxupquote{class }}\sphinxcode{\sphinxupquote{client.}}\sphinxbfcode{\sphinxupquote{ObjectTracker}}}{\emph{feed}, \emph{transmitter}, \emph{tracker='kcf'}, \emph{mode='watch'}}{}
Object tracker is a thread that has an opencv tracker

The tracker is given access to a live feed where gets updated frames and to a transmitter which can send commands to a picar. Object tracking with the PiCar is poor, likely due to the instability of the head, paticularly when moving.
\index{deselect() (client.ObjectTracker method)@\spxentry{deselect()}\spxextra{client.ObjectTracker method}}

\begin{fulllineitems}
\phantomsection\label{\detokenize{index:client.ObjectTracker.deselect}}\pysiglinewithargsret{\sphinxbfcode{\sphinxupquote{deselect}}}{}{}
Clear selected object and stop tracking

\end{fulllineitems}

\index{run() (client.ObjectTracker method)@\spxentry{run()}\spxextra{client.ObjectTracker method}}

\begin{fulllineitems}
\phantomsection\label{\detokenize{index:client.ObjectTracker.run}}\pysiglinewithargsret{\sphinxbfcode{\sphinxupquote{run}}}{}{}
Track object in background using latest frame

Updates the box location for the viewer. Also sends command using transmitter so the car will look towards the object thus centering it in the screen.

\end{fulllineitems}

\index{select() (client.ObjectTracker method)@\spxentry{select()}\spxextra{client.ObjectTracker method}}

\begin{fulllineitems}
\phantomsection\label{\detokenize{index:client.ObjectTracker.select}}\pysiglinewithargsret{\sphinxbfcode{\sphinxupquote{select}}}{\emph{frame}}{}
Select object to track from a frame

Gives the user the ability to select an object from a paused frame.
The selected contains the portion of the image which will be tracked.

\end{fulllineitems}

\index{track() (client.ObjectTracker method)@\spxentry{track()}\spxextra{client.ObjectTracker method}}

\begin{fulllineitems}
\phantomsection\label{\detokenize{index:client.ObjectTracker.track}}\pysiglinewithargsret{\sphinxbfcode{\sphinxupquote{track}}}{\emph{frame}}{}
Get a single update of the tracker.

\end{fulllineitems}


\end{fulllineitems}

\index{Transmitter (class in client)@\spxentry{Transmitter}\spxextra{class in client}}

\begin{fulllineitems}
\phantomsection\label{\detokenize{index:client.Transmitter}}\pysiglinewithargsret{\sphinxbfcode{\sphinxupquote{class }}\sphinxcode{\sphinxupquote{client.}}\sphinxbfcode{\sphinxupquote{Transmitter}}}{\emph{addr='192.168.0.212'}, \emph{port=5000}}{}
Class containing a socket and the ability to send commands to the server

There should only be one instance unless multiple cars are being controlled.
Every controller must be passed the transmitter in order to work.
\index{send\_cmd() (client.Transmitter method)@\spxentry{send\_cmd()}\spxextra{client.Transmitter method}}

\begin{fulllineitems}
\phantomsection\label{\detokenize{index:client.Transmitter.send_cmd}}\pysiglinewithargsret{\sphinxbfcode{\sphinxupquote{send\_cmd}}}{\emph{cmd}}{}
Send command to picar reciever

\end{fulllineitems}


\end{fulllineitems}

\index{voice\_command() (in module client)@\spxentry{voice\_command()}\spxextra{in module client}}

\begin{fulllineitems}
\phantomsection\label{\detokenize{index:client.voice_command}}\pysiglinewithargsret{\sphinxcode{\sphinxupquote{client.}}\sphinxbfcode{\sphinxupquote{voice\_command}}}{}{}
Get input from the user using voice recognition.

Voice commands are limited because implimenting a call for each can become complicated. The car has a tendency to get out of control, because the user cannot respond quickly enough.

\end{fulllineitems}



\renewcommand{\indexname}{Python Module Index}
\begin{sphinxtheindex}
\let\bigletter\sphinxstyleindexlettergroup
\bigletter{c}
\item\relax\sphinxstyleindexentry{client}\sphinxstyleindexpageref{index:\detokenize{module-client}}
\indexspace
\bigletter{p}
\item\relax\sphinxstyleindexentry{picar}\sphinxstyleindexpageref{index:\detokenize{module-picar}}
\indexspace
\bigletter{s}
\item\relax\sphinxstyleindexentry{server}\sphinxstyleindexpageref{index:\detokenize{module-server}}
\end{sphinxtheindex}

\renewcommand{\indexname}{Index}
\printindex
\end{document}